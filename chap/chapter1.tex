\section{The premise}

This document aims to create an entry point to unreal engine VR development. Hereby the focus is not necessarily on creating a full blown game mechanic, but rather on projects, which would be both achievable and beneficial in a scientific context.\\
Although this document aims to create sort of a general overview, it is being written specifically for the \textit{Institute of data processing}(IPE) at the KIT. The IPE has already conducted multiple VR projects. I will use them as an example to explain, what purpose a VR project could have in the scientific context:

\begin{itemize}
\item One project used the CAD models for the \textit{KATRIN} [CITATION] experiment and imported those as models into a VR setting, which can be used present the experiment to a broader public, which is not able to visit the restricted area for example.
\item Part of the research at the KIT campus north is the recording of x ray tomographies. Using this method researchers were able to create detailed 3D models of insects within fossils. There has been an effort to import these insect models into a VR setting. The capabilities of the unreal engine could help to animate the movement and behaviour of these long extinct species for example. An audience would be able to experience these animals first hand.
\end{itemize}

\section{document overview}

This document aims to achieve several things:

\begin{description}
\item[entry point] This document should provide a basic overview of the first steps to be taken, when starting VR development with the unreal engine, such as the installation of required programs and the setup of the needed hardware.
\item[choices] The success of a VR project is significantly determined, by the planning process or the general "vision" for the final project. There are some game-specific and also VR-specific design choices, that have to be well thought about when determining the initial goal of the project.
\item[tutorial] I will attempt to provide tutorial-like sections for chosen topics, which will be helpful for developing a first application scaffold.
\item[literature] The unreal engine is too big of a topic to be explained well in a single document. Thus, this document will not provide a "real" tutorial. Instead I will reference some literature and other resources, that have done a good job of explaining certain aspects of the topic.
\end{description}

