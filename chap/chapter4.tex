\section{chapter overview}

in this section I will introduce some helpful resources in the form of books and websites, which will be helpful for the development of VR project with the unreal engine.

% Include a small summary of all the following sections, so that 
% a reader can quickly skim over, which ressource might be the 
% most interesting at the moment

\begin{description}
\item[Unreal for Mobile and Standalone VR \cite{MobileVR}] The focus of this book is specifically on development for mobile VR hardware, although all the topics described are also applicable for \textit{tethered VR platforms}\footnote{Tethered VR platforms are those platforms, which need mounted basestations to track the position of the user. These platforms have the advantage of offering a 6 DOF (degree of freedom) VR experience in contrast to the 3 DOF of mobile VR  platforms}.\\
This book includes general advice for the setup of a development pipeline, tools and design choices. Furthermore it includes two in depth tutorials: One for developing an interactive product showcase and another for developing a full game with all essential game mechanics.
\end{description}

\section{Book \textit{Unreal for Mobile and Standalone VR}}

This book can be acquired for free from the KIT library\cite{KitBib}, when accessed from the KIT wifi or a VPN, using the following URL:

\url{https://link.springer.com/book/10.1007%2F978-1-4842-4360-2}

\setlength{\fboxsep}{0pt}
\setlength{\fboxrule}{0pt}
\begin{figure}[h]
\centering
\fbox{\includegraphics[width=0.6\textwidth]{./fig/cover_unreal_mobile.png}}
\caption[Unreal for Mobile and Standalone VR]{cover of the book "Unreal for Mobile and Standalone VR"}
\label{fig:unrealmobilecover}
\end{figure}

\paragraph{Chapter 3: Before you start}

The third chapter from the book was the basis for the chapter \ref{chap:designchoices} about design choices in this document. Everything mentioned and a lot more is described here in detail.\\
Furthermore the third chapter includes some nice tips on \textit{low hanging fruits}\footnote{the term describes aspects of a project, which can improve the project a lot and are easy to achieve} of every VR project. 

\paragraph{Chapter 4: Unreal for VR}

The main focus of this chapter is on creating a VR project for the \textit{Oculus Go} platform specifically. The basic tutorial for setting up simple motion controller locomotion blueprints is generally applicable nonetheless.

\paragraph{chapter 6: Creating an Interactive VR Presentation}

This chapter is a full on tutorial on how to create a VR product presentation. Since a product presentation has a lot in common with a presentation for a science project, this chapter is the most relevant to get started with a project.\\
The chapter roughly contains the following content:
\begin{itemize}
\item Pipeline for creating the scene aesthetics using Unreal and Blender
\item lighting the scene
\item basic environment interactions with floating UI buttons and sequencer animations
\end{itemize}

