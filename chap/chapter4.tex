\section{chapter overview}

in this section I will introduce some helpful resources in the form of books and websites, which will be helpful for the development of VR project with the unreal engine.

% Include a small summary of all the following sections, so that 
% a reader can quickly skim over, which ressource might be the 
% most interesting at the moment

\begin{description}
\item[Unreal for Mobile and Standalone VR] The focus of this book is specifically on development for mobile VR hardware, although all the topics described are also applicable for \textit{tethered VR platforms}\footnote{Tethered VR platforms are those platforms, which need mounted basestations to track the position of the user. These platforms have the advantage of offering a 6 DOF (degree of freedom) VR experience in contrast to the 3 DOF of mobile VR  platforms}.\\
This book includes general advice for the setup of a development pipeline, tools and design choices. Furthermore it includes two in depth tutorials: One for developing an interactive product showcase and another for developing a full game with all essential game mechanics.
\end{description}

\section{Book "Unreal for Mobile and Standalone VR"}

This book can be acquired for free from the KIT library\cite{KitBib}, when accessed from the KIT wifi or a VPN.

\setlength{\fboxsep}{0pt}
\setlength{\fboxrule}{0pt}
\begin{figure}[h]
\centering
\fbox{\includegraphics[width=0.6\textwidth]{./fig/cover_unreal_mobile.png}}
\caption[Unreal for Mobile and Standalone VR]{cover of the book "Unreal for Mobile and Standalone VR"}
\label{fig:unrealgoodspecs}
\end{figure}
