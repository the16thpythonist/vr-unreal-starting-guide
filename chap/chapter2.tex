\section{Installing the development environment}

\subsection{Unreal engine requirements}

The \textit{minimal requirements}\footnote{\url{https://docs.unrealengine.com/en-US/GettingStarted/RecommendedSpecifications/index.html}} for installing the unreal engine are quite low, as can be seen in figure \ref{fig:unrealminspecs} \cite{UnrealEngineSpecs}. These requirements make it seem as if development with the unreal engine was actually possible on such a machine.\\
I have tried to install the development environment on a computer, which just barely satisfied these requirements. From the experience I can say, that while the program actually starts, it is completely unusable. The graphics card and the CPU will be absolutely overwhelmed.

\setlength{\fboxsep}{0pt}
\setlength{\fboxrule}{0pt}
\begin{figure}[h]
\centering
\fbox{\includegraphics[width=\textwidth]{./fig/unreal_specs_minimal.png}}
\caption[Minimal system requirements unreal engine]{Minimal system requirements for the unreal engine}
\label{fig:unrealminspecs}
\end{figure}

On the far bottom of the same requirements page, epic games lists the specs of the PC's, which themselves use for their game development, as shown in figure \ref{fig:unrealgoodspecs}.\\
These should be considered as the \textit{actual} minimal requirements for a smooth development experience.

\setlength{\fboxsep}{0pt}
\setlength{\fboxrule}{0pt}
\begin{figure}[h]
\centering
\fbox{\includegraphics[width=0.4\textwidth]{./fig/unreal_specs_good.png}}
\caption[Reasonable system requirements unreal engine]{Reasonable system requirements for the unreal engine}
\label{fig:unrealgoodspecs}
\end{figure}

In the end, a machine with the above or comparable hardware specifications \textit{should} be used to guarantee a lag free experience of the unreal editor software.

\section{Installing the unreal engine}

pass