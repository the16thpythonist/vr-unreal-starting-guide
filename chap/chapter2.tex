\section{Installing the development environment}

\subsection{Unreal engine requirements}

The \textit{minimal requirements}\footnote{\url{https://docs.unrealengine.com/en-US/GettingStarted/RecommendedSpecifications/index.html}} for installing the unreal engine are quite low, as can be seen in figure \ref{fig:unrealminspecs} \cite{UnrealEngineSpecs}. These requirements make it seem as if development with the unreal engine was actually possible on such a machine.\\
I have tried to install the development environment on a computer, which just barely satisfied these requirements. From the experience I can say, that while the program actually starts, it is completely unusable. The graphics card and the CPU will be absolutely overwhelmed.

\setlength{\fboxsep}{0pt}
\setlength{\fboxrule}{0pt}
\begin{figure}[h]
\centering
\fbox{\includegraphics[width=\textwidth]{./fig/unreal_specs_minimal.png}}
\caption[Minimal system requirements unreal engine]{Minimal system requirements for the unreal engine}
\label{fig:unrealminspecs}
\end{figure}

On the far bottom of the same requirements page, epic games lists the specs of the PC's, which themselves use for their game development, as shown in figure \ref{fig:unrealgoodspecs}.\\
These should be considered as the \textit{actual} minimal requirements for a smooth development experience.

\setlength{\fboxsep}{0pt}
\setlength{\fboxrule}{0pt}
\begin{figure}[h]
\centering
\fbox{\includegraphics[width=0.4\textwidth]{./fig/unreal_specs_good.png}}
\caption[Reasonable system requirements unreal engine]{Reasonable system requirements for the unreal engine}
\label{fig:unrealgoodspecs}
\end{figure}

In the end, a machine with the above or comparable hardware specifications \textit{should} be used to guarantee a lag free experience of the unreal editor software.

\section{Installing the unreal engine}

The installation of the unreal engine is managed by installing the \textit{Epic Games Launcher}. This launcher is used to manage and launch games, developed by Epic Games. But it can also be used to download the Unreal Engine from. Using the Epic Games Launcher provides the benefit, that it is easily possible to download multiple different versions of the engine. All these versions can be installed and managed on the same computer at once.\\
The installation of the unreal engine is rather well documented on the following web page:

\url{https://docs.unrealengine.com/en-US/GettingStarted/Installation/index.html}

\section{Additional programs}

When creating a game, a developer has to deal with many different things:
\begin{itemize}
\item Creating the 3D models for the characters and the environment
\item Creating surface material textures for the 3D models
\item Creating animation sequences
\end{itemize}

Just to name a few.\\
While all these steps can be done within the Unreal Editor itself, it is strongly discouraged to do so. All of these processes have their own, specialized application programs. These programs provide a superior environment with better functionality.\\
The problem is, that most of these programs require a license fee.

\section{Installing Blender}

\textit{Blender}\cite{Blender} is an open-source program, which is mainly used for 3D modeling and VFX animations sequences. But it can also be used for the other operations. Even though for some steps Blender is also not the best tool to be used, it is still better than using the built in functionality of the Unreal Engine.

Blender is installed by downloading an installer from the following page:

\url{https://www.blender.org/download/}

Running the installer program should install the blender program, which will be executable through a desktop icon.