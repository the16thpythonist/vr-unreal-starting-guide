% Here I will say a few words about why it is so important
% to keep the scope of the project as small as possible, so that 
% it doesnt get shitty along the way
\section{The scope of the project}

\subsection{basic considerations}

Before starting to work on a project, several things have to be outlined at first. One of the major questions leading up to a VR project will be "What will it have to accomplish?" or "Why are we doing this?". A likely answer for the case of a VR project in a scientific context would be to showcase some sort of scientific project or scientific paper to a broader public. The VR space is used as a medium to create a superficial understanding within those people, which cannot grasp the concrete details.\\
The next question should be about the scope of the project "What specific aspects of our work would we like to showcase?". This includes the task of figuring out, which features the VR presentation has to have, but more importantly which features it \textit{does not} have to implement. It is important to understand, that it is a \textit{very hard and time consuming} task to design and implement a \textit{good} VR experience. Thats why a reasonably step during the development is to explicitly think about, where complexity can be reduced as much as possible. This goes for all aspects of a game. Possible approaches would be to think about at which points it would not hurt the aesthetics to use easier textures or how to strategically limit the playable area.\\

\subsection{example 1}

To further explain these considerations, I will give a fictional example:\\
Biologists have discovered a fascinating new method of transportation within an exotic tadpole species. Since their movement cannot accurately be described with images alone, they want to create a VR game to showcase their findings to the public. More concretely they define the goal of the VR presentation as \textit{Explaining the steps involved in the movement of the tadpoles}.\\
During an initial brainstorming it is quickly clear, that the setting of the game should be underwater, where the user can observe the tadpoles natural motion in 3D. A lot of ideas are gathered, for example the following ones:

\begin{itemize}
\item The playable area could be designed like the bottom of the ocean, or the inside if a pond or aquarium
\item There could be many tadpoles everywhere, swimming in different directions
\item A user interface could be used to switch between an external view of the tadpoles to an internal one, with all the muscles.
\item The user will be sitting in a submarine capsule which can be controlled to swim within the world
\end{itemize}

While all these ideas provide great inspiration and would undoubtably make a nice VR experience, it is important to keep in mind, that usually a lot of people invest a lot of time into developing nice-looking, smooth game experiences. Since creating a presentation for the public usually is a rather low priority within a scientific project, the amount of time and man power, which can be spent on the VR presentation is very limited.\\
Due to the lack of time the team stripped down the features of the final brainstorming to this final description of the game:
The player will find itself standing in a bright white room. The player can navigate the room by walking into all directions. In the center of the room there is a circular platform. In front of the platform is a single button menu, which can be used with the motion controllers. The buttons are "spawn tadpole" and "description". Upon pressing the "spawn tadpole" button a tadpole will appear floating stationary above the platform, performing its signature motion. Upon pressing description an audio description of the tadpoles movement will start to play. The player can investigate the motion from all different sides, by moving around the platform, but he cannot leave the room.\\
This is a good example of tightening the scope of a project to save in production effort. A lot of features were sacrificed for the sake of simplicity, yet the original goal is still achieved.

% In this section I will be telling about that chapter from the 
% book about mobile VR I read. It will be about what design 
% choices can be made for VR specifically.
\section{VR specific design choices}

When embracing a VR project there are some elements of game design especially relevant to the platform being virtual reality specifically.\\
In the following sections I will be summarizing key aspects, which are listed in the following book:

[INSERT BOOK REFERENCE]

For further details on the topics, read the corresponding chapters of the book.

\subsection{Why VR?}

This is usually the first and most important question that has to be asked before starting a VR project: \textit{Do we actually need VR for this}. We will look at the example of creating a virtual reality showcase of some scientific experiment or paper.\\ 
The aim of such a project is of course to create a visually appealing presentation for an audience. But there are multiple different ways of presenting a complex topic to someone: Beginning with just static illustrations on a placard, a narrated slideshow, a simple animation without interactions... And all of these can probably be realized with less effort, than a VR experience would require. Every medium has its unique advantages. In that sense, it would only be worth the additional effort if the given presentation would benefit greatly from the advantages of VR. These advantages are:

\begin{itemize}
\item An unrivaled sense of 3 dimensional \textit{depth} and \textit{size}
\item Interactivity
\end{itemize}

So the questions leading up to whether or not VR is to be used as the medium of presentation are:

\begin{itemize}
\item Does our project need to display size and depth differences in 3 dimensions?
\item Does our project benefit from incorporating interactivity
\end{itemize}

If the scope for example is to illustrate a (mostly) 2 dimensional process without any kind of user interaction, VR might not be the right choice.\\
Going even further VR might even be a very \textit{wrong} choice in some situations: Consider the following example: A team has approximately X hours to spent on creating a presentation of their topic. During this time they could either create an above average presentation with traditional media such as placards and animations \textit{or} a below average VR presentation. Chances are high, that the audience will be put off by the fact that the VR is "bad" quality. On the other hand, the audience might have enjoyed the good traditional media, even though it is \textit{"just"} 2 dimensional.

Knowing this, the question \textit{Why VR?} provides us with good hints of how to create a successful VR experience: Designing a simple, yet highly interactive application structure for the user. As well as creating an environment which makes use of close range \textit{and} medium range objects to nicely incorporate a feeling of 3 dimensional scale and depth.

\subsection{The method of locomotion}

The method of locomotion has been one of the most important aspects of VR game development.\\
In a traditional game, the user will usually play a character, whose direct translatory movement can be controlled using analog control sticks on a game controller. Simply adapting this method for VR games has led to many users experiencing so called \textit{motion sickness}, or more generally \textit{simulation sickness}. This is most likely due to the fact, that the brain is experiencing movement through the visual stimulus, which it cannot confirm with its sense of acceleration of balance of the bodily stimulus. These different stimuli cause the user to experience a feeling of disorientation and general sickness.

This circumstance has led to the development of alternative locomotion mechanics: Among the most popular ones are \textit{teleportation} and \textit{rail based locomotion}.\\
With teleportation the user will utilize the motion controllers to point to a specific place in the world. After pressing an additional button, the vision of the player is quickly faded out into darkness and then faded in again, with the new position being the one previously pointed at.\\
Rail based locomotion still uses direct translatory movement of the player position, but with an important change in game design: The player is confined in a vessel such as a wagon or an aircraft, which is moving be itself. This type of motion is easier for the brain, as it can relate to real world experiences such as using a car or train, to explain the differences in stimuli.

That all being said, there is strong evidence, that direct character motion using analog sticks might still be a valid possibility. Studies could show, that even though the majority of people first exposed to a VR environment experience simulation sickness, it fades away after regular exposure. There is only a small fraction of the population, whose brain cannot adapt to the experience even after a long exposure time.

But since one cannot expect the audience to have accustomed to the motion sickness yet, it might still be a good idea to implement the teleportation option for movement.