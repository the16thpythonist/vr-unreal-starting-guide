\documentclass{include/thesisclass}
% Main File - Based on thesisclass.cls
% Comments are mostly in English
% ------------------------------------------------------------------------------
% Further files in folder:
%  - include/cmds.tex (for macros and additional commands)
%  - include/kitlogo.pdf (for titlepage)
%  - lit.bib (bibtex bibliography database)
%  - include/titlepage.tex (for layout of titelpage)
% ------------------------------------------------------------------------------
% Useful Supplied Packages:
% amsmath, amssymb, mathtools, bbm, upgreek, nicefrac,
% siunitx, varioref, booktabs, graphicx, tikz, multicol

%% -------------------------
%% |    Thesis Settings    |
%% -------------------------
% english or ngerman (new german für neue deutsche Rechtschreibung statt german)
\SelectLanguage{english}

% -----------------------
% DETAILS OF THE DOCUMENT
% -----------------------
\newcommand{\thesisauthor}{Jonas Teufel}
\newcommand{\thesistopic}{-}
\newcommand{\thesisentopic}{A Guide for future VR development at the IPE with the Unreal Engine}
\newcommand{\thesislongtopic}{Very long and very detailed description of the very interesting thesis topic (only necessary for pdfsubject tag).}
\newcommand{\thesisinstitute}{Institut für Prozessdatenverarbeitung \\
und Elektronik}
\newcommand{\thesisreviewerone}{Andreas Kopmann}
\newcommand{\thesisreviewertwo}{Prof. Dr. E. Vil}
\newcommand{\thesisadvisorone}{} % to use: enter names and uncomment in titlepg
\newcommand{\thesisadvisortwo}{}
\newcommand{\thesistimestart}{01.04.2015} % on titlepage
\newcommand{\thesistimeend}{30.09.2015} % on titlepage
\newcommand{\thesistimehandin}{30.09.2015} % on second page 'preamble'
\newcommand{\thesispagehead}{Internship report: UCAPhantom Plugin} % page heading

%% ---------------------
%% |    PDF - Setup    |
%% ---------------------
% This information will appear embed into the PDF file as meta data, but will 
% not be printed anywhere
\hypersetup
{
    pdfauthor={\thesisauthor},
    pdftitle={Internship report: \thesistopic},
    pdfsubject={\thesislongtopic},
    pdfkeywords={kit,ipe,bachelor,internship,phantom camera,libuca,\thesisauthor}
}

%% --------------------------------------
%% |    Settings for Word Separation    |
%% --------------------------------------
% Help for separation:
% In German package the following hints are additionally available:
% "- = Additional separation
% "| = Suppress ligation and possible separation (e.g. Schaf"|fell)
% "~ = Hyphenation without separation (e.g. bergauf und "~ab)
% "= = Hyphenation with separation before and after
% "" = Separation without a hyphenation (e.g. und/""oder)

% Describe separation hints here:
\hyphenation
{
    über-nom-me-nen an-ge-ge-be-nen
    %Pro-to-koll-in-stan-zen
    %Ma-na-ge-ment  Netz-werk-ele-men-ten
    %Netz-werk Netz-werk-re-ser-vie-rung
    %Netz-werk-adap-ter Fein-ju-stier-ung
    %Da-ten-strom-spe-zi-fi-ka-tion Pa-ket-rumpf
    %Kon-troll-in-stanz
}

\definecolor{dkgreen}{rgb}{0,0.6,0}
\definecolor{gray}{rgb}{0.5,0.5,0.5}
\definecolor{mauve}{rgb}{0.58,0,0.82}

\lstset{frame=none,
  language=Java,
  aboveskip=3mm,
  belowskip=3mm,
  showstringspaces=false,
  columns=flexible,
  basicstyle={\small\ttfamily},
  numbers=none,
  numberstyle=\tiny\color{gray},
  keywordstyle=\color{blue},
  commentstyle=\color{dkgreen},
  stringstyle=\color{mauve},
  breaklines=true,
  breakatwhitespace=true,
  tabsize=3
}

\lstset{frame=none,
  language=C,
  aboveskip=4mm,
  belowskip=2mm,
  showstringspaces=false,
  columns=flexible,
  basicstyle={\small\ttfamily},
  numbers=none,
  numberstyle=\tiny\color{gray},
  keywordstyle=\color{blue},
  commentstyle=\color{dkgreen},
  stringstyle=\color{mauve},
  breaklines=true,
  breakatwhitespace=true,
  tabsize=3
}


%% -----------------------
%% |    Main Document    |
%% -----------------------
\usepackage{lipsum} % for Lorem Ipsum text example


\begin{document}
    % Titlepage and ToC
    \FrontMatter

    \input{include/titlepage}
    \input{include/preamble}

    \begingroup \let\clearpage\relax    % in order to avoid listoffigures and
    \tableofcontents                    % listoftables on new pages
    \listoffigures
    \listoftables
    \endgroup
    \cleardoublepage
    
    % Contents
    \MainMatter

    \chapter{Introduction}
    \section{The premise}

This document aims to create an entry point to unreal engine VR development. Hereby the focus is not necessarily on creating a full blown game mechanic, but rather on projects, which would be both achievable and beneficial in a scientific context.\\
Although this document aims to create sort of a general overview, it is being written specifically for the \textit{Institute of data processing}(IPE) at the KIT. The IPE has already conducted multiple VR projects. I will use them as an example to explain, what purpose a VR project could have in the scientific context:

\begin{itemize}
\item One project used the CAD models for the \textit{KATRIN} [CITATION] experiment and imported those as models into a VR setting, which can be used present the experiment to a broader public, which is not able to visit the restricted area for example.
\item Part of the research at the KIT campus north is the recording of x ray tomographies. Using this method researchers were able to create detailed 3D models of insects within fossils. There has been an effort to import these insect models into a VR setting. The capabilities of the unreal engine could help to animate the movement and behaviour of these long extinct species for example. An audience would be able to experience these animals first hand.
\end{itemize}

\section{document overview}

This document aims to achieve several things:

\begin{description}
\item[entry point] This document should provide a basic overview of the first steps to be taken, when starting VR development with the unreal engine, such as the installation of required programs and the setup of the needed hardware.
\item[choices] The success of a VR project is significantly determined, by the planning process or the general "vision" for the final project. There are some game-specific and also VR-specific design choices, that have to be well thought about when determining the initial goal of the project.
\item[tutorial] I will attempt to provide tutorial-like sections for chosen topics, which will be helpful for developing a first application scaffold.
\item[literature] The unreal engine is too big of a topic to be explained well in a single document. Thus, this document will not provide a "real" tutorial. Instead I will reference some literature and other resources, that have done a good job of explaining certain aspects of the topic.
\end{description}


    
    \chapter{First steps}
    \section{Installing the development environment}

\subsection{Unreal engine requirements}

The \textit{minimal requirements}\footnote{\url{https://docs.unrealengine.com/en-US/GettingStarted/RecommendedSpecifications/index.html}} for installing the unreal engine are quite low, as can be seen in figure \ref{fig:unrealminspecs} \cite{UnrealEngineSpecs}. These requirements make it seem as if development with the unreal engine was actually possible on such a machine.\\
I have tried to install the development environment on a computer, which just barely satisfied these requirements. From the experience I can say, that while the program actually starts, it is completely unusable. The graphics card and the CPU will be absolutely overwhelmed.

\setlength{\fboxsep}{0pt}
\setlength{\fboxrule}{0pt}
\begin{figure}[h]
\centering
\fbox{\includegraphics[width=\textwidth]{./fig/unreal_specs_minimal.png}}
\caption[Minimal system requirements unreal engine]{Minimal system requirements for the unreal engine}
\label{fig:unrealminspecs}
\end{figure}

On the far bottom of the same requirements page, epic games lists the specs of the PC's, which themselves use for their game development, as shown in figure \ref{fig:unrealgoodspecs}.\\
These should be considered as the \textit{actual} minimal requirements for a smooth development experience.

\setlength{\fboxsep}{0pt}
\setlength{\fboxrule}{0pt}
\begin{figure}[h]
\centering
\fbox{\includegraphics[width=0.4\textwidth]{./fig/unreal_specs_good.png}}
\caption[Reasonable system requirements unreal engine]{Reasonable system requirements for the unreal engine}
\label{fig:unrealgoodspecs}
\end{figure}

In the end, a machine with the above or comparable hardware specifications \textit{should} be used to guarantee a lag free experience of the unreal editor software.

\section{Installing the unreal engine}

pass
    
    \chapter{Design Choices}
    % Here I will say a few words about why it is so important
% to keep the scope of the project as small as possible, so that 
% it doesnt get shitty along the way
\section{The scope of the project}

\subsection{basic considerations}

Before starting to work on a project, several things have to be outlined at first. One of the major questions leading up to a VR project will be "What will it have to accomplish?" or "Why are we doing this?". A likely answer for the case of a VR project in a scientific context would be to showcase some sort of scientific project or scientific paper to a broader public. The VR space is used as a medium to create a superficial understanding within those people, which cannot grasp the concrete details.\\
The next question should be about the scope of the project "What specific aspects of our work would we like to showcase?". This includes the task of figuring out, which features the VR presentation has to have, but more importantly which features it \textit{does not} have to implement. It is important to understand, that it is a \textit{very hard and time consuming} task to design and implement a \textit{good} VR experience. Thats why a reasonably step during the development is to explicitly think about, where complexity can be reduced as much as possible. This goes for all aspects of a game. Possible approaches would be to think about at which points it would not hurt the aesthetics to use easier textures or how to strategically limit the playable area.\\

\subsection{example 1}

To further explain these considerations, I will give a fictional example:\\
Biologists have discovered a fascinating new method of transportation within an exotic tadpole species. Since their movement cannot accurately be described with images alone, they want to create a VR game to showcase their findings to the public. More concretely they define the goal of the VR presentation as \textit{Explaining the steps involved in the movement of the tadpoles}.\\
During an initial brainstorming it is quickly clear, that the setting of the game should be underwater, where the user can observe the tadpoles natural motion in 3D. A lot of ideas are gathered, for example the following ones:

\begin{itemize}
\item The playable area could be designed like the bottom of the ocean, or the inside if a pond or aquarium
\item There could be many tadpoles everywhere, swimming in different directions
\item A user interface could be used to switch between an external view of the tadpoles to an internal one, with all the muscles.
\item The user will be sitting in a submarine capsule which can be controlled to swim within the world
\end{itemize}

While all these ideas provide great inspiration and would undoubtably make a nice VR experience, it is important to keep in mind, that usually a lot of people invest a lot of time into developing nice-looking, smooth game experiences. Since creating a presentation for the public usually is a rather low priority within a scientific project, the amount of time and man power, which can be spent on the VR presentation is very limited.\\
Due to the lack of time the team stripped down the features of the final brainstorming to this final description of the game:
The player will find itself standing in a bright white room. The player can navigate the room by walking into all directions. In the center of the room there is a circular platform. In front of the platform is a single button menu, which can be used with the motion controllers. The buttons are "spawn tadpole" and "description". Upon pressing the "spawn tadpole" button a tadpole will appear floating stationary above the platform, performing its signature motion. Upon pressing description an audio description of the tadpoles movement will start to play. The player can investigate the motion from all different sides, by moving around the platform, but he cannot leave the room.\\
This is a good example of tightening the scope of a project to save in production effort. A lot of features were sacrificed for the sake of simplicity, yet the original goal is still achieved.

% In this section I will be telling about that chapter from the 
% book about mobile VR I read. It will be about what design 
% choices can be made for VR specifically.
\section{VR specific design choices}

When embracing a VR project there are some elements of game design especially relevant to the platform being virtual reality specifically.\\
In the following sections I will be summarizing key aspects, which are listed in the following book:

\textit{Unreal for Mobile and Standalone VR}\cite{MobileVR}

For further details on the topics, read the corresponding chapters of the book.

\subsection{Why VR?}

This is usually the first and most important question that has to be asked before starting a VR project: \textit{Do we actually need VR for this}. We will look at the example of creating a virtual reality showcase of some scientific experiment or paper.\\ 
The aim of such a project is of course to create a visually appealing presentation for an audience. But there are multiple different ways of presenting a complex topic to someone: Beginning with just static illustrations on a placard, a narrated slideshow, a simple animation without interactions... And all of these can probably be realized with less effort, than a VR experience would require. Every medium has its unique advantages. In that sense, it would only be worth the additional effort if the given presentation would benefit greatly from the advantages of VR. These advantages are:

\begin{itemize}
\item An unrivaled sense of 3 dimensional \textit{depth} and \textit{size}
\item Interactivity
\end{itemize}

So the questions leading up to whether or not VR is to be used as the medium of presentation are:

\begin{itemize}
\item Does our project need to display size and depth differences in 3 dimensions?
\item Does our project benefit from incorporating interactivity
\end{itemize}

If the scope for example is to illustrate a (mostly) 2 dimensional process without any kind of user interaction, VR might not be the right choice.\\
Going even further VR might even be a very \textit{wrong} choice in some situations: Consider the following example: A team has approximately X hours to spent on creating a presentation of their topic. During this time they could either create an above average presentation with traditional media such as placards and animations \textit{or} a below average VR presentation. Chances are high, that the audience will be put off by the fact that the VR is "bad" quality. On the other hand, the audience might have enjoyed the good traditional media, even though it is \textit{"just"} 2 dimensional.

Knowing this, the question \textit{Why VR?} provides us with good hints of how to create a successful VR experience: Designing a simple, yet highly interactive application structure for the user. As well as creating an environment which makes use of close range \textit{and} medium range objects to nicely incorporate a feeling of 3 dimensional scale and depth.

\subsection{The method of locomotion}

The method of locomotion has been one of the most important aspects of VR game development.\\
In a traditional game, the user will usually play a character, whose direct translatory movement can be controlled using analog control sticks on a game controller. Simply adapting this method for VR games has led to many users experiencing so called \textit{motion sickness}, or more generally \textit{simulation sickness}. This is most likely due to the fact, that the brain is experiencing movement through the visual stimulus, which it cannot confirm with its sense of acceleration of balance of the bodily stimulus. These different stimuli cause the user to experience a feeling of disorientation and general sickness.

This circumstance has led to the development of alternative locomotion mechanics: Among the most popular ones are \textit{teleportation} and \textit{rail based locomotion}.\\
With teleportation the user will utilize the motion controllers to point to a specific place in the world. After pressing an additional button, the vision of the player is quickly faded out into darkness and then faded in again, with the new position being the one previously pointed at.\\
Rail based locomotion still uses direct translatory movement of the player position, but with an important change in game design: The player is confined in a vessel such as a wagon or an aircraft, which is moving be itself. This type of motion is easier for the brain, as it can relate to real world experiences such as using a car or train, to explain the differences in stimuli.

That all being said, there is strong evidence, that direct character motion using analog sticks might still be a valid possibility. Studies could show, that even though the majority of people first exposed to a VR environment experience simulation sickness, it fades away after regular exposure. There is only a small fraction of the population, whose brain cannot adapt to the experience even after a long exposure time.

But since one cannot expect the audience to have accustomed to the motion sickness yet, it might still be a good idea to implement the teleportation option for movement.
    
    \chapter{Literature}
    \section{chapter overview}

in this section I will introduce some helpful resources in the form of books and websites, which will be helpful for the development of VR project with the unreal engine.

% Include a small summary of all the following sections, so that 
% a reader can quickly skim over, which ressource might be the 
% most interesting at the moment

\begin{description}
\item[Unreal for Mobile and Standalone VR] The focus of this book is specifically on development for mobile VR hardware, although all the topics described are also applicable for \textit{tethered VR platforms}\footnote{Tethered VR platforms are those platforms, which need mounted basestations to track the position of the user. These platforms have the advantage of offering a 6 DOF (degree of freedom) VR experience in contrast to the 3 DOF of mobile VR  platforms}.\\
This book includes general advice for the setup of a development pipeline, tools and design choices. Furthermore it includes two in depth tutorials: One for developing an interactive product showcase and another for developing a full game with all essential game mechanics.
\end{description}

\section{Book "Unreal for Mobile and Standalone VR"}

This book can be acquired for free from the KIT library\cite{KitBib}, when accessed from the KIT wifi or a VPN.

\setlength{\fboxsep}{0pt}
\setlength{\fboxrule}{0pt}
\begin{figure}[h]
\centering
\fbox{\includegraphics[width=0.6\textwidth]{./fig/cover_unreal_mobile.png}}
\caption[Unreal for Mobile and Standalone VR]{cover of the book "Unreal for Mobile and Standalone VR"}
\label{fig:unrealgoodspecs}
\end{figure}

    
	\chapter{Tutorials}
	
% What Would I write here? What is worth to be made a 
% tutorial?
% + Setting up the HTC Vive
% + Fundamentals about how unreal works with all its blueprints etc.
% - Basic Locomotion system
% + How the sequence works with animations etc.

%\section{Setting up the HTC Vive}

% The images for this I will need to add, when I am at the IPE
%How to set up the hardware and the software

%\section{Fundamentals of Unreal}

%Write something about how the engine generally works.

\section{Using the Sequencer for Basic Animations}

\paragraph{Basic Introduction}

Since Version 4.22 the Unreal Engine has been extended with the \textit{Sequencer} Functionality.\\
Within the editor, objects are being represented by their position, which is their X, Y and Z coordinates. Additionally to their position objects also have \textit{properties}. You can think of a light source for example, one of its properties would be the intensity of the light.\\
With the sequencer tool you can create new "sequence" objects. With these sequences you can define specific sequences of position and property changes for objects within the game world. What would be an example for this? Imagine a game setting, where there is a closed door in a hallway. As the player walks down the hallway, he will trigger a new Sequence called "DoorSequence" for example. Within this sequence, you have previously defined, that the door will rotate outwards from its angles by 90 degrees, within a time frame of 1 second. Upon triggering the sequence the player will witness the animation of the door opening automatically, until it does not move anymore after 1 second and then remain open.\\
To summerize: The sequencer tool can be used for simple, world-bound animations. Why only simple animations? The sequencer system is relatively limited to manipulating "only" the position and other properties of \textit{whole} objects. Thus it is not able to for example deform whole objects. But when animating a characters facial expressions for example, what you want to do is to make a series of deformations of the actual model as time progresses. To make these kind of \textit{more dynamic} animations a different functionality of the Unreal Engine has to be used.

\paragraph{Basic concepts of sequences}

The most important concept for working with animations is that of a \textit{keyframe}. Overall you can compare the sequence object with a video: In the end the animation itself is nothing more than a series of still images played rapidly in series. Each one of these still images within a video is called \textit{frame}. As such the different moments during a sequence are also called frame.\\
Now imagine an example: You simply want to move a red cube from left to right. So what you would do is create a first keyframe at the 0 seconds mark within the sequence, with the cube still being on the left. Then you would create another keyframe at 1 second, with the cube being on the right, where you want it to be at the end. Than you could already play the sequence and see an animation. Now you could ask: "How come, that I see the whole animation? I havent yet defined all the positions of the cube \textit{in between} the start and the end!". This is why sequences are so convenient: You only need to set the \textit{defining moments} or \textit{keyframes} of an animation. The program will \textit{interpolate} all the frames in between those to moments to create a smooth animation.

So in the end working with the sequencer really just comes down to setting the appropriate keyframes and letting the program do the rest

\paragraph{Setting up a Sequence}

Now we are getting to the practical part. First, a quick summary of what has to be done for setting up a new sequence in the Unreal Engine:
\begin{itemize}
\item Setting
\end{itemize}
	
    %\emptychapter[3]{ROOT Routines}     % usage: \emptychapter[page displayed 
                                        %        in toc]{name of the chapter}
    %\chapter{Conclusions}
    %\section{Summary}

% What do I even want to say about it?
% - Programming the cli was a very good, way to get to know the 
% camera protocol, especially programming the mock, where I tried 
% to reverse engineer the cameras inner workings
% - The mock really helped the development process, as some of the 
% testing did not have to be done with the camera fully set up.
% - A quick out of the box tool to test the most common operations 
% with the camera, which does neither rely on using the telnet 
% connection and memorizing the protocol or compiling the whole 
% libuca pipeling
\subsection{PhantomCLI command line tools}

The PhantomCLI\footnote{Available at \url{https://github.com/the16thpythonist/phantom-cli}} package provides simple command line tools to perform basic interactions with the phantom camera. Furthermore a simple camera mock simulation was implemented, which will emulate certain camera behaviours on a local machine.\\
First of all implementing these command line programs was very helpful to understand the PH16 camera protocol a lot better. Using a high level programming language such as Python enabled much more rapid exploratory approach for working with the camera, than the direct low-level implementation in C would have provided. Among all the functionalities of the PhantomCLI package, the mock was most helpful to understand the inner workings of the camera. By attempting to emulate camera behaviour, I was able to gain a lot of insights. Among the most important insights were the concrete structures of the transfer encodings and their packing into the raw ethernet frames, which was not fully clear by protocol description alone.\\
Beyond the learning, experience the PhantomCLI package provides a handy tool for working with the camera. The mock script provided the possibility to test certain features of the uca-phantom plugin without having to have access to the full camera setup. The getter and setter commands were useful to check and adjust camera settings on the go. The command line tools even provide functionality, which the final LibUCA plugin does not. It is advised to use the PhantomCLI utilities to change the \textit{binning mode} of the camera, as the required restart would not be handled well by the plugin.

% What do I even want to say about it?
% - It does what it is supposed to do:
%	- The trigger modes are implemeted
%	- It has a non-memread mode for viewing current frames
%	- It has memread mode for read out of large image chunks
% - It matches the speed requirement using the 10G line way better 
% than the shitty 1G transmission.
% - It works with concert
\subsection{UcaPhantom plugin}

The UcaPhantom\footnote{Available at \url{https://github.com/the16thpythonist/uca-phantom}} package is a plugin for the UCA Library. It is an interfacing application, which connects the phantom line of ultra fast cameras with the UCA library, which is being used for tomography experiments at the KIT.\\
The plugin implements the required functions of a LibUCA extension: The start and end of a recording, the readout of frames and different trigger modes to start an image acquisition. Above the basic requirements the plugin implements the \textit{memread} mode, which utilizes the 10G connectivity provided by the phantom camera. The memread mode features fast reception of raw ethernet frames and decoding to provide a transmission speed of roughly $600 \frac{\text{MB}}{\text{s}}$.

% So what are the things, that are still ugly?

% PhantomCLI
% - The mock is far away from being an accurate representation 
% of the camera. What would be needed is a sort of internal 
% representation of camera state, as opposed to an input output 
% maschine right now.
% - The currently implemented features only are a small subset of 
% all the possible commands

% UcaPhantom
% - A lot of utility features are missing. Like fancy schmancy 
% filter and delay settings for the external trigger or 
% configuration of the axuliary ports
% - Changing modes is not implemented and pretty much a pain. 
% This has to be done using the phantomCLI and then a restart
% - Error handling system is still not as good as it could be. 
% Doing things in slightly the wrong order etc. would crash the 
% program. It is not really robust.
% Integration testing with 
% - There are not tests, that confirm it working with another 
% phantom camera, while in theory it should work with every 
% camera that implements Visio researches new protocol PH16
\section{Future work}

For both the PhantomCLI command line tools as well as the UcaPhantom plugin future improvements could be made.\par 

The PhantomCLI package currently only implements a small subset of PH16 protocol used to communicate with the phantom camera. It would be possible to implement more detailed and specific configuration commands as well as the reception of timestamp data for example.\\
The mock would also be a good aspect for future improvements. Currently the mock is running in its own process, but it is more of a input-output system, than an actual simulation. It does maintain an internal state of the camera with most of its properties, but this is only static. The properties dont have an effect on the actual functionality on the return of an image for example. To build a more functional mock, it would be required to implement a dynamic internal model of the camera, which reacts to state changes and which is also able to start a recording with an internal frame buffer for example.\par 

Like the PhantomCLI package, the UcaPhantom plugin also only implements the most important subset of the PH16 protocol functionality. Here Functions such as the retrieval of timestamps or the in-application configuration of auxiliary ports could be implemented.\\
Another issue is that the software implementation is not as robust as it could be. The application will sometimes crash, when it is used beyond its intended use cases. In these cases catching the error and displaying a warning message might me appropriate.\\
Due to the lack of another phantom camera model during development, it is unclear whether the application works with other models as well. While the application should work with all newer models utilizing the PH16 protocol, it would be good to confirm this with actual tests.

	
	\chapter*{Changelog}	
	
\section*{Version 0}

\paragraph{0.0.0 - 18.11.2019}

\begin{itemize}
\item Created the git repository
\item Copied the template for the KIT bachelors degree to be used as the basis of this document
\item Added a first version of the "premise" chapter and "document overview"
\item Added chapter about the system requirements of the unreal engine.
\end{itemize}

\paragraph{0.0.1 - 19.11.2019}

\begin{itemize}
\item Added the chapter about installing unreal engine
\item Added the chapter about the other application
\end{itemize}

Possible improvements in the future:

\begin{itemize}
\item Extend the other application chapter with the list of all the applications that was given in the book
\end{itemize}

\paragraph{0.0.2 - 21.11.2019} 

\begin{itemize}
\item Added a draft of the chapter about design choices. The chapter
\begin{itemize}
\item Section about the scope of the project: Rather make the scope smaller, because a VR game is a lot of work
\item Section about "Why VR?" why should you even make a VR project. Could traditional media do the same?
\item Section about the method of locomotion in VR. Why teleportation might be a good idea.
\end{itemize}
\item Began working on the "ressources" chapter, which is supposed to list a few helpful ressources (books mainly) in the pursuit of making a VR representation
\end{itemize}

\paragraph{0.0.3 - 22.11.2019}

\begin{itemize}
\item Wrote the chapter about the book "Unreal for mobile and standalone VR"
\item removed the appendix section for now. Since I dont need that yet.
\item removed the inclusion of the preamble.tex
\item removed the inclusion of the list of tables for now
\item Added preliminary sections for the "tutorials" section.
\end{itemize}

% What has to be done for the first version of this document
% - Wirte the tutorials, that I wanted to write
% - Fix the missing citations and stuff in the other chapters

\paragraph{0.0.4 - 26.11.2019}

\begin{itemize}
\item Added all the missing citations
\item Decided, that for the first version there will only be a sequencer section in the tutorials chapter
\item Started to write the sequencer tutorial
\end{itemize}

% for the next version
% - Finish writing the sequencer tutorial
% - Add images to the sequencer tutorial

	
	
    % appendix for more or less interesting calculations
    \Appendix
    \chapter*{\appendixname} \addcontentsline{toc}{chapter}{\appendixname}
    % to make the appendix appear in ToC without number. \appendixname = 
    % Appendix or Anhang (depending on chosen language)
    \input{./chap/appendix.tex} %\cleardoublepage

    % Bibliography
    \TheBibliography

    % BIBTEX
    % use if you want citations to appear even if they are not referenced to: 
    % \nocite{*} or maybe \nocite{Kon64,And59} for specific entries
    %\nocite{*}
    \bibliographystyle{babalpha}
    
    \bibliography{lit}

    % THEBIBLIOGRAPHY
    %\begin{thebibliography}{000}
    %    \bibitem{ident}Entry into Bibliography.
    %\end{thebibliography}
\end{document}
